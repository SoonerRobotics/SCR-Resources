% Sooner Competitive Robotics Constitution
%
% Template from https://gist.github.com/eyliu/166080

\documentclass[12pt]{cls/constitution}
\usepackage{mathpazo}
\begin{document}
\title{Sooner Competitive Robotics Constitution}
\author{Sooner Competitive Robotics}
\date{Last Modified on May 2nd, 2021}
\maketitle
\setcounter{tocdepth}{0}
\tableofcontents
\newpage

\article{Name}

The name of this organization shall be Sooner Competitive Robotics, hereinafter “the Organization.”

\article{Purpose}

The purpose of the Organization is to provide a centralized hub for the funding and support of OU's robotics competition teams, ensuring the availability of financial and technical resources, managerial accountability, and ultimate success at competitions. 

\article{Membership}

\section{}
All currently enrolled students at the University of Oklahoma are eligible for membership. 

\section{}
Membership in, association with, and benefits emanating from the Organization and its related activities shall be based upon such considerations as performance, educational achievement, and other criteria related to strong leadership and technical competence. Judgments in this regard shall not be based solely on an individual’s race, color, religion, national origin, age, gender, sexual orientation, disability, veteran status, marital status, or political beliefs. Further, the purpose of the organization must be consistent with public policy as established by prevailing University Community standards.  

\article{Officers \& Advisor}

\section{}
The Officers of the Organization shall be President, Vice President, Treasurer, Secretary, and Lab Manager.
The President and Vice President shall appoint all Team Captains; in cases of disagreement, the President has final say.
Subteam leads may be appointed or removed as decided by their Team Captain with the approval of the Officer team.

\section{}
To be considered a part of Voting Membership, a member must be a current active member of the organization, and request the Officer team to be included in the Voting Membership. The Officer team may decide the specific criteria each year. A member must ensure they are included in the Voting Membership each year, as this status does not automatically persist.

\section{}
The Officers of the Organization shall be elected by the Voting Membership.

\section{}
Any Officer, Team Captain, or Advisor may be removed from their position by a unanimous vote of the other Officers. Any Officer or Advisor may be removed by a three-fourths vote of those in attendance at an Officer Impeachment meeting. A petition may be signed by at least one-third of the Voting Membership and brought to the President or Vice President, which shall result in an Officer Impeachment meeting no more than two weeks later. The meeting time shall be decided by the Voting Membership and the Officers.

\section{}
The President is the chief executive officer of the Organization, and will be the point of contact with the University. The President shall preside at all meetings and direct the affairs of the Organization with the advice and consent of the other Officers. The President shall decide for which competitions the Organization will support teams based on the Organization's current capabilities. The President shall ensure that the goals of leadership and technical learning are met by all. In the event that there is no current President, the remainder of the President’s term shall be filled in the following order of succession:  Vice President, then Treasurer, then Secretary, then a qualified student of the Advisor’s choice.

\section{}
The Vice President is in charge of recruitment and outreach efforts, including aiding the Treasurer in searching for sponsors. As such, the Vice President should primarily be responsible for upkeep of a "Sponsorship Packet" containing updated information, photos, and statistics to share with potential donors. The Vice President shall also aid the President in administrative efforts, and shall preside at meetings in the absense of the President.

\section{}
The Treasurer is responsible for keeping record of the financial status of the Organization. This includes obtaining receipts from other members and updating the general ledger of the Organization. The Treasurer will also seek out and apply for sponsorships from companies, the University, and other potential sources of funding for the Organization. The treasurer will collect dues from members (if applicable).

\section{}
The Secretary is responsible for the public face of the Organization. This includes regularly taking photos and videos at meetings (or eliciting them from other members) and posting to the Organization's social media platforms. The Secretary will also maintain and send updates over the Organization's email list, and keep track of contact information for alumni and active membership. The Secretary will take minutes at each meeting.

\section{}
The Lab Manager is responsible for maintaining the lab space and any other resources used primarily by the Organization. This involves regularly cleaning, keeping parts and tools organized and put away, and restocking general lab supplies. Supplies commonly ordered that should be restocked will be discussed by the Officers and Team Captains, and any member may make suggestions for additions to the list. The Lab Manager will also use the remainder of funds at the end of the academic year for large purchases for the lab, with approval of the Officers.

\section{}
The Organization shall have as an Advisor a full-time member of the University faculty or staff. This role must be filled in order to register with the University.

\section{}
The Advisor’s role in the Organization shall be limited to offering advice or feedback to the Organization, in addition to other duties delegated by the President.

\section{}
The Advisor will continue to serve in this role until they choose to abdicate, or if a majority of the Officers vote to remove the Advisor. Another eligible faculty or staff member may be appointed by the Officers in their place with a simple majority vote.

\article{Election Procedure}

\section{}
Elections shall not be valid without substantial compliance with this Article. 

\section{}
Elections shall be held near the end of the Spring Semester, before the start of final exam week. In the event that there are no candidates for an office, the office may remain open until a special election is called. 

\section{}
Elections may only take place if at least one week’s notice of the date, time, and location of the election is given to members of the Organization's Voting Membership. This includes general Spring elections, and any special elections for a position that was left open, abdicated, or removed.

\section{}
Voting shall be anonymous and secret. A "Vote of No Confidence" shall always be offered. If "No Confidence" receives a simple majority vote, the position will remain open until a special election is called.

\section{}
Voting shall be done by first-past-the-post voting. The voting system may be changed with a unanimous vote of the Officers.

\section{}
The Secretary shall document the names of the Voting Membership present at the election and the results of each vote. A transition meeting of new and old Officers should be held no more than two weeks after general Spring elections. At the end of such meeting, new electees officially become Officers.

\section{}
When more than one office is being elected, the order of voting shall proceed as follows: President, then Vice-President, then Treasurer, then Secretary, then Lab Manager.

\section{}
Any member of the Voting Membership of the Organization may run for any office, or nominate another member of the Voting Membership to run. A nominated member may accept or decline.

\section{}
A member of the Voting Membership may be a candidate for more than one office, but shall be immediately removed from consideration for any other Officer position after being elected. In no instance may one person fulfill more than one Officer position. 

\article{Additional Governing Principles}

\section{}
The Organization is subject to Local, State, and Federal Laws.

\section{}
The Organization is independent of any national or parent organization.

\section{}
As the Organization is not affiliated with any parent organization, this Constitution is the sole governing document, subject to the terms of the University of Oklahoma Student Code.

\article{Amendment Procedure}
This Constitution may be amended by the Officers with a unainmous vote, but this may be overturned by a two-thirds vote of the Organization’s Voting Membership present at a general meeting. 

\end{document}
